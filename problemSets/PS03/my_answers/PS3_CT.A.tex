\documentclass[12pt,letterpaper]{article}
\usepackage{graphicx,textcomp}
\usepackage{natbib}
\usepackage{setspace}
\usepackage{fullpage}
\usepackage{color}
\usepackage[reqno]{amsmath}
\usepackage{amsthm}
\usepackage{fancyvrb}
\usepackage{amssymb,enumerate}
\usepackage[all]{xy}
\usepackage{endnotes}
\usepackage{lscape}
\newtheorem{com}{Comment}
\usepackage{float}
\usepackage{hyperref}
\newtheorem{lem} {Lemma}
\newtheorem{prop}{Proposition}
\newtheorem{thm}{Theorem}
\newtheorem{defn}{Definition}
\newtheorem{cor}{Corollary}
\newtheorem{obs}{Observation}
\usepackage[compact]{titlesec}
\usepackage{dcolumn}
\usepackage{tikz}
\usetikzlibrary{arrows}
\usepackage{multirow}
\usepackage{xcolor}
\newcolumntype{.}{D{.}{.}{-1}}
\newcolumntype{d}[1]{D{.}{.}{#1}}
\definecolor{light-gray}{gray}{0.65}
\usepackage{url}
\usepackage{listings}
\usepackage{color}

\definecolor{codegreen}{rgb}{0,0.6,0}
\definecolor{codegray}{rgb}{0.5,0.5,0.5}
\definecolor{codepurple}{rgb}{0.58,0,0.82}
\definecolor{backcolour}{rgb}{0.95,0.95,0.92}

\lstdefinestyle{mystyle}{
	backgroundcolor=\color{backcolour},   
	commentstyle=\color{codegreen},
	keywordstyle=\color{magenta},
	numberstyle=\tiny\color{codegray},
	stringstyle=\color{codepurple},
	basicstyle=\footnotesize,
	breakatwhitespace=false,         
	breaklines=true,                 
	captionpos=b,                    
	keepspaces=true,                 
	numbers=left,                    
	numbersep=5pt,                  
	showspaces=false,                
	showstringspaces=false,
	showtabs=false,                  
	tabsize=2
}
\lstset{style=mystyle}
\newcommand{\Sref}[1]{Section~\ref{#1}}
\newtheorem{hyp}{Hypothesis}

\title{Problem Set 3}
\date{Due: November 11, 2024}
\author{Applied Stats/Quant Methods 1}


\begin{document}
	\maketitle
	\section*{Instructions}
	\begin{itemize}
		\item Please show your work! You may lose points by simply writing in the answer. If the problem requires you to execute commands in \texttt{R}, please include the code you used to get your answers. Please also include the \texttt{.R} file that contains your code. If you are not sure if work needs to be shown for a particular problem, please ask.
	\item Your homework should be submitted electronically on GitHub.
	\item This problem set is due before 23:59 on Sunday November 11, 2024. No late assignments will be accepted.

	\end{itemize}

		\vspace{.25cm}
	
\noindent In this problem set, you will run several regressions and create an add variable plot (see the lecture slides) in \texttt{R} using the \texttt{incumbents\_subset.csv} dataset. Include all of your code.

	\vspace{.5cm}
\section*{Question 1}
\vspace{.25cm}
\noindent We are interested in knowing how the difference in campaign spending between incumbent and challenger affects the incumbent's vote share. 
	\begin{enumerate}
		\item Run a regression where the outcome variable is \texttt{voteshare} and the explanatory variable is \texttt{difflog}.	
		\vspace{.5cm}
		
			\lstinputlisting[language=R, firstline=45, lastline=69]{PS3_CT.A.R} 
		So the regression coefficient is:\\
		$$
		\hat{\beta}_0 = 0.57903071 \text{ (intercept)}
		$$
		$$
		\hat{\beta}_1 = 0.04166632 \text{ (slope of difflog)}
		$$
		\vspace{.5cm}
		
		\item Make a scatterplot of the two variables and add the regression line. 	
		\vspace{.5cm}
		
		\lstinputlisting[language=R, firstline=74, lastline=81]{PS3_CT.A.R} 
		  \begin{center}
			\includegraphics[width=0.7\linewidth]{"scatter_plot1"}
		\end{center}
		
		\item Save the residuals of the model in a separate object.	
		\vspace{.5cm}
		
		\lstinputlisting[language=R, firstline=86, lastline=96]{PS3_CT.A.R} 
		\vspace{5cm}
		
		\item Write the prediction equation.\\
		
		\vspace{.5cm}
		Based on the calculated regression coefficients, the prediction equation is:
		
		$$
		\hat{voteshare} = \hat{\beta}_0 + \hat{\beta}_1 \times difflog
		$$
		
		The specific values are:
		
		$$
		\hat{voteshare} = 0.57903071 + 0.04166632 \times difflog
		$$
		
	\end{enumerate}
	
\newpage

\section*{Question 2}
\noindent We are interested in knowing how the difference between incumbent and challenger's spending and the vote share of the presidential candidate of the incumbent's party are related.	\vspace{.25cm}
	\begin{enumerate}
		\item Run a regression where the outcome variable is \texttt{presvote} and the explanatory variable is \texttt{difflog}.	
		
		\vspace{.5cm}
		\lstinputlisting[language=R, firstline=106, lastline=130]{PS3_CT.A.R} 
			So the regression coefficient is:\\
		$$
		\hat{\beta}_0 = 0.50758333 \text{ (intercept)}
		$$
		$$
		\hat{\beta}_1 = 0.02383723 \text{ (slope of difflog)}
		$$
		\vspace{.5cm}
		
		\item Make a scatterplot of the two variables and add the regression line. 	
		
		\vspace{.5cm}
		\lstinputlisting[language=R, firstline=135, lastline=142]{PS3_CT.A.R} 
		  \begin{center}
			\includegraphics[width=0.7\linewidth]{"scatter_plot2"}
		\end{center}
		
		\item Save the residuals of the model in a separate object.	
		
		\vspace{.5cm}
		\lstinputlisting[language=R, firstline=147, lastline=157]{PS3_CT.A.R} 
		\vspace{.5cm}
		
		\item Write the prediction equation.
		
		\vspace{.5cm}
		Based on the calculated regression coefficients, the prediction equation is:
		
		$$
		\hat{voteshare} = \hat{\beta}_0 + \hat{\beta}_1 \times difflog
		$$
		
		The specific values are:
		
		$$
		\hat{voteshare} = 0.50758333 + 0.02383723 \times difflog
		$$
		
	\end{enumerate}
	
	\newpage	
\section*{Question 3}

\noindent We are interested in knowing how the vote share of the presidential candidate of the incumbent's party is associated with the incumbent's electoral success.
	\vspace{.25cm}
	\begin{enumerate}
		\item Run a regression where the outcome variable is \texttt{voteshare} and the explanatory variable is \texttt{presvote}.
		
		\vspace{.5cm}
		\lstinputlisting[language=R, firstline=166, lastline=190]{PS3_CT.A.R} 
		So the regression coefficient is:\\
		$$
		\hat{\beta}_0 = 0.4413299 \text{ (intercept)}
		$$
		$$
		\hat{\beta}_1 = 0.3880184 \text{ (slope of presvote)}
		$$
		\vspace{.5cm}
		
		\item Make a scatterplot of the two variables and add the regression line. 
			
	    	\vspace{.5cm}
			\lstinputlisting[language=R, firstline=196, lastline=202]{PS3_CT.A.R} 
			\begin{center}
				\includegraphics[width=0.7\linewidth]{"scatter_plot3"}
			\end{center}
			
		\item Write the prediction equation.
		
		\vspace{.5cm}
		Based on the calculated regression coefficients, the prediction equation is:
		
		$$
		\hat{voteshare} = \hat{\beta}_0 + \hat{\beta}_1 \times difflog
		$$
		
		The specific values are:
		
		$$
		\hat{voteshare} = 0.4413299 + 0.3880184 \times difflog
		$$
		
	\end{enumerate}
	

\newpage	
\section*{Question 4}
\noindent The residuals from part (a) tell us how much of the variation in \texttt{voteshare} is $not$ explained by the difference in spending between incumbent and challenger. The residuals in part (b) tell us how much of the variation in \texttt{presvote} is $not$ explained by the difference in spending between incumbent and challenger in the district.
	\begin{enumerate}
		\item Run a regression where the outcome variable is the residuals from Question 1 and the explanatory variable is the residuals from Question 2.	
		
		\vspace{.5cm}
		\lstinputlisting[language=R, firstline=211, lastline=239]{PS3_CT.A.R} 
		So the regression coefficient is:\\
		$$
		\hat{\beta}_0 = 5.326175\times 10^{-17}  \text{ (intercept)}
		$$
		$$
		\hat{\beta}_1 = 0.3880184 \text{ (slope of presvote)}
		$$
		\vspace{.5cm}
		
		\item Make a scatterplot of the two residuals and add the regression line. 	\
		
		\vspace{.5cm}
		\lstinputlisting[language=R, firstline=244, lastline=251]{PS3_CT.A.R} 
		\begin{center}
			\includegraphics[width=0.7\linewidth]{"scatter_plot4"}
		\end{center}
		
		\item Write the prediction equation.
		
		Based on the calculated regression coefficients, the prediction equation is:
		$$
		\hat{e}_{model1} = \hat{\beta}_0 + \hat{\beta}_1 \times e_{model2}
		$$
		
		With specific values:
		$$
		\hat{e}_{model1} = 5.326175 \times 10^{-17} + 0.2568770 \times e_{model2}
		$$
		
		After simplification:
		$$
		\hat{e}_{model1} \approx 0 + 0.2568770 \times e_{model2}
		$$
		
		Therefore:
		$$
		\hat{e}_{model1} = 0.2568770 \times e_{model2}
		$$
		
	\end{enumerate}
	
	\newpage	

\section*{Question 5}
\noindent What if the incumbent's vote share is affected by both the president's popularity and the difference in spending between incumbent and challenger? 
	\begin{enumerate}
		\item Run a regression where the outcome variable is the incumbent's \texttt{voteshare} and the explanatory variables are \texttt{difflog} and \texttt{presvote}.	
		
		\vspace{.5cm}
		\lstinputlisting[language=R, firstline=260, lastline=285]{PS3_CT.A.R} 
		So the regression coefficient is:\\
		$$
		\hat{\beta}_0 = 0.44864422 \text{ (intercept)}
		$$
		$$
		\hat{\beta}_1 = 0.03554309 \text{ (slope of difflog)}
		$$
		$$
		\hat{\beta}_2 = 0.25687701 \text{ (slope of presvote)}
		$$
		\vspace{.5cm}
		
		\item Write the prediction equation.	
		
		\vspace{.5cm}
		Based on the calculated regression coefficients, the prediction equation is:
		$$
		\hat{voteshare} = \hat{\beta}_0 + \hat{\beta}_1 \times difflog + \hat{\beta}_2 \times presvote
		$$
		
		With specific values:
		$$
		\hat{voteshare} = 0.44864422 + 0.03554309 \times difflog + 0.25687701 \times presvote
		$$
		
		\item What is it in this output that is identical to the output in Question 4? Why do you think this is the case?
		
		\vspace{.5cm}
		Here's the English translation and LaTeX code:\\
		In regression model of Question 5, which includes two independent variables difflog and presvote, we obtained the regression coefficient for presvote:\\
		$$
		\hat{\beta}_2 = 0.25687701
		$$
		This is identical to the regression coefficient of presvote (0.256877) in Question 4.\\
		Explanation:\\
		According to the properties of multiple linear regression, when we include multiple independent variables in a regression model, each variable's regression coefficient reflects its independent effect on the dependent variable after controlling for other independent variables. In Question 4, we effectively regressed residuals\_model1 (voteshare residuals after controlling for difflog) on residuals\_model2 (presvote residuals after controlling for difflog), where the regression coefficient reflected the effect of presvote on voteshare after controlling for difflog.\\
		Therefore, in Question 5, when we simultaneously regress voteshare on difflog and presvote, the regression coefficient of presvote (0.25687701) is exactly the same as its effect coefficient on voteshare in Question 4. This is because in multiple regression models, presvote's regression coefficient represents its independent effect on voteshare after controlling for difflog, consistent with the results from the residual regression in Question 4.
		
	\end{enumerate}




\end{document}
